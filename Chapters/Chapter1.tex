\chapter{Vectors}
\section{Introduction}
A real number can be represented by a point on a line, which is a 2-dimensional space, $\R$
\begin{center}
\begin{tikzpicture}
\draw[->] (0,0) -- (6,0);
\newcommand{\offset}{0.1}
\foreach \x in {0,...,5}
	\draw (\x,\offset) -- (\x,-\offset);

\foreach \x in {0,...,3}
		\draw (\x,-0.1) node[anchor=north] {\x};
\draw (4.5,-0.1) node[anchor=north] {4.5};	
\draw (4.5,0) node[] {x};	
\end{tikzpicture}
\end{center}

\noindent a pair of real numbers can be represented by a point on a plane, which is a 2-dimensional space, $\R^2$
\begin{center}
\begin{tikzpicture}[scale=0.5]
axis
	\draw[->] (-0.5,0) -- coordinate (x axis mid) (6,0) node[anchor=west]{$x_1$};
    	\draw[->] (0,-0.5) -- coordinate (y axis mid) (0,6) node[anchor=south]{$x_2$};;
    	%ticks
	\newcommand{\offset}{0.1}

    	\foreach \x in {0,...,5}
     		\draw (\x,\offset) -- (\x,-\offset);
    	
\foreach \y in {0,...,5}
     		\draw (\offset,\y) -- (-\offset,\y);
\draw[fill=black] (4,2) circle (0.05) node[anchor=west]{$\colvec{2}{4}{2}$};
\end{tikzpicture}
\end{center}

\noindent a triplet of real numbers can be represented by a point in 3D space, $\R^3$
\begin{center}
\begin{tikzpicture}[scale=0.6]
\node [] (0) at (0, 0) {};
\node [] (1) at (0, 4) {};
\node [] (2) at (4, 0) {};
\node [] (3) at (-2, -3) {};
\node [] (4) at (2, 2) {};
\node [] (5) at (2, 0) {};
\node [] (6) at (-1, -1.5) {};
\node [] (7) at (1, -1.5) {};
\node [] (8) at (-1, 0.5) {};
\node [] (9) at (0, 2) {};
\node [] (10) at (1, 0.5) {};
\draw [->] (0.center) to (1.center) node[anchor=south]{$x_2$};
\draw [->] (0.center) to (2.center) node[anchor=west]{$x_1$};
\draw [->] (0.center) to (3.center) node[anchor=north east]{$x_3$};
\draw [dashed] (6.center) to (7.center);
\draw [dashed] (7.center) to (5.center);
\draw [dashed] (5.center) to (4.center);
\draw [dashed] (4.center) to (9.center);
\draw [dashed] (9.center) to (8.center);
\draw [dashed] (8.center) to (6.center);
\draw [dashed] (8.center) to (10.center);
\draw [dashed] (10.center) to (4.center);
\draw [dashed] (10.center) to (7.center);
\draw[fill=black](10.center) circle (0.1);
\end{tikzpicture}
\end{center}


\begin{definition}
	A vector is an ordered collection of $n$ numbers
\end{definition}

\begin{notation}
Usually vectors are given by letters, such as $u,v,w$. In textbooks vectors are written with bold font. In handwriting vectors are often written with a right arrow on top, such as $\overrightarrow{u}$. We will underline vectors, like so: $\underline{u}$.	
\end{notation}

\begin{definition}
Let us consider vector $\ul{u}\in\R^n$. The $i-$th component of vector \[\ul{u}=\colvec{3}{u_1}{\vdots}{u_n}\] is $u_i$
\end{definition}

\begin{example}
\[\ul{u}=\colvec{3}{3}{7}{11}\in\R^3\Rightarrow u_1=3, u_2=7, u_3=11 \]	
\end{example}

\begin{definition}
Let us consider vectors $\ul{u} \in\R^n$ and $\ul{v} \in\R^n$. Vector $\ul{w} \in\R^n$ is a sum of $\ul{u}$ and $\ul{v}$, $\ul{w}=\ul{u}+\ul{v}$, if $w_i=u_i+v_i$ for all $i=1,\dots,n$
\end{definition}

\begin{example}
\[\ul{u}=\colvec{3}{3}{5}{1}, \ul{v}=\colvec{3}{-1}{0}{1}, \ul{w}=\ul{u}+\ul{v}=\colvec{3}{3+(-1)}{5+0}{1+1}=\colvec{3}{2}{5}{2}\]	
\end{example}

\begin{example}
\[\ul{u}=\colvec{3}{3}{9}{-2}, \hspace{2mm} \ul{v}=\colvec{4}{1}{2}{3}{0}\] $\uv+\vv$ is \textbf{not} defined: both vectors should have the same number of components.
\end{example}

\begin{definition}
\begin{enumerate}
	\item Vectors $\uv\in\R^n$ and $\vv\in\R^n$ are equal, if $u_i=v_i$ for all $i=1,\dots,n$
	\item A scalar is just another name for real number
	\item Let us consider a scalar $\alpha\in\R$ and vector $\uv \in\R^n$. A product of $\alpha$ and $\uv$ is defined as: \[\alpha\uv = \alpha\cdot\colvec{3}{u_1}{\vdots}{u_n} = \colvec{3}{\alpha\cdot u_1}{\vdots}{\alpha\cdot u_n}\]
\end{enumerate}
\end{definition}

\begin{example}
\[\alpha = 3, \uv = \colvec{4}{-1}{2}{5}{7}\Rightarrow \alpha\cdot\uv\colvec{4}{3\cdot -1}{3\cdot 2}{3\cdot 5}{3\cdot 7} = \colvec{4}{-3}{6}{15}{21} \]	
\end{example}

\begin{definition}
	Let us consider scalars $\alpha$ and $\beta$, and vectors $\uv\in\R^n$ and $\vv\in\R^n$. A sum of $\alpha\cdot\uv + \beta\cdot\vv$ is called a linear combination of vectors $\uv$ and $\vv$.
\end{definition}

\begin{example}
\[2\cdot \colvec{3}{-1}{3}{5} + 3\cdot \colvec{3}{7}{2}{1} + 5\cdot \colvec{3}{1}{0}{-1} = \colvec{3}{24}{12}{8}\]	
\end{example}

\begin{example}
\[\uv - \vv = 1\cdot \uv +(-1)\cdot \vv = \colvec{3}{u_1-v_1}{\vdots}{u_i-v_i}\]
\end{example}

\begin{note}
\[\uv - \uv = \colvec{3}{u_1-u_1}{\vdots}{u_i-u_i}=\ul{0} \]
\end{note}

\begin{definition}
	Vector $\uv\in\R^n$ is called a zero vector if all $u_i = 0$, $i=1,\dots,n$. The zero vector is often written as $\ul{0}\in\R^n$
\end{definition}

\section{Vector Representations and Operations}
A vector can be represented in the following ways 
\begin{enumerate}
	\item As ordered collection of numbers, for example $\uv = \colvec{2}{3}{5}$
	\item As an arrow in space \begin{center}
\begin{tikzpicture}[scale=0.7]
	\draw[->] (-0.5,0) -- coordinate (x axis mid) (4,0) node[anchor=west]{$x_1$};
    	\draw[->] (0,-0.5) -- coordinate (y axis mid) (0,4) node[anchor=south]{$x_2$};;
    	%ticks
	\newcommand{\offset}{0.1}

    	\foreach \x in {0,...,3}
     		\draw (\x,\offset) -- (\x,-\offset);
    	
\foreach \y in {0,...,3}
     		\draw (\offset,\y) -- (-\offset,\y);

\draw[->] (0,0)--(3,1)node[anchor=west]{$\uv = \colvec{2}{3}{1}$};
\end{tikzpicture}

	\end{center}

	\item As a point in space, the endpoint of a vector from the origin. 
\end{enumerate}

\subsection{Combination of Vectors}
Let us consider vectors $\uv = \colvec{2}{3}{1}$, $\vv = \colvec{2}{-1}{2}$ and $\wv = \uv + \vv =\colvec{2}{2}{3}$. The following picture would represent the vectors
\begin{center}
\begin{tikzpicture}
\draw[->] (-2.5,0) -- coordinate (x axis mid) (4,0) node[anchor=west]{$x_1$};
\draw[->] (0,0) -- coordinate (y axis mid) (0,4) node[anchor=south]{$x_2$};;
    	%ticks
	\newcommand{\offset}{0.1}

    	\foreach \x in {-2,...,3}{
     		\draw (\x,\offset) -- (\x,-\offset);
		\draw (\x,-0.1) node[anchor=north] {\x};
		}
		
    	
\foreach \y in {0,...,3}{
     		\draw (\offset,\y) -- (-\offset,\y);
		\ifthenelse{ \equal{\y}{0} }{}{
		\draw (-0.1,\y) node[anchor=east] {\y};
		}
		}

\draw[dashed] (2,3)--(3,1);
\draw[dashed] (-1,2)--(2,3);
\draw[->] (0,0)--(3,1)node[anchor=west]{$\uv$};
\draw[->] (0,0)--(2,3)node[anchor=west]{$\wv$};
\draw[->] (0,0)--(-1,2)node[anchor=east]{$\vv$};
\end{tikzpicture}
\end{center}

\begin{example}
Let us consider vector $\uv = \colvec{2}{3}{1}$. \\ \\
\begin{enumerate}
\item What is $2\cdot \uv$? \\ 
 We can calculate it as follows:
\[
2\cdot \uv = 2\cdot\colvec{2}{3}{1} = \colvec{2}{6}{2}
\]
We stretch vector $\uv$ two times along the line defined by vector $\uv$ as follows: \\
\begin{center}
\begin{tikzpicture}[scale=0.75]
\draw[->] (-2,0) -- coordinate (x axis mid) (8,0) node[anchor=west]{$x_1$};
\draw[->] (0,-0.5) -- coordinate (y axis mid) (0,4) node[anchor=south]{$x_2$};;
    	%ticks

% 2u
\draw[->,thick] (0,0)--(6,2);

% u
\draw[->,very thick] (0,0)--(3,1);

\draw (3, 1) node[anchor=south] {$\uv$};
\draw (6, 2) node[anchor=south] {$2\cdot\uv$};
\end{tikzpicture}
\end{center}

\item What is $-\uv$? \\
Simply reverse the direction of $\uv$ as follows:
\begin{center}
\begin{tikzpicture}
\draw[->] (-2,0) -- coordinate (x axis mid) (8,0) node[anchor=west]{$x_1$};
\draw[->] (0,-0.5) -- coordinate (y axis mid) (0,4) node[anchor=south]{$x_2$};;
    	%ticks

% vector u
\draw[->,very thick] (0,0)--(3,1);

% vector -u
\draw[<-,very thick] (0,0)--(3,1);

% label u
\draw (3, 1) node[anchor=south] {$\uv$};

% label -u
\draw (.5,-.5) node[anchor=south] {$-\uv$};

\end{tikzpicture}

\end{center}

\item What will be the representation of $\alpha\uv$, for all possible values of $\alpha$? \\
An endless \textcolor{red}{line} as follows \\

\begin{center}
\begin{tikzpicture}[scale=0.75]
\draw[->] (-2,0) -- coordinate (x axis mid) (8,0) node[anchor=west]{$x_1$};
\draw[->] (0,-0.5) -- coordinate (y axis mid) (0,4) node[anchor=south]{$x_2$};;
    	%ticks

\draw[->,very thick] (0,0)--(3,1);
\draw (3, 1) node[anchor=south] {$\uv$};
\draw[red] (-3,-1)--(9,3);
\end{tikzpicture}
\end{center}

\end{enumerate}

\end{example}
\begin{example}
Let us consider two vectors $\uv\in\R^2$ and $\vv\in\R^2$. What will be the representation of all linear combinations of $\uv$ and $\vv$, that is what will be $\alpha\uv+\beta\vv$, for all $\alpha$ and $\beta$?
\begin{enumerate}
\item A \textbf{plane}, if $\uv \neq \ul{0}$ and $\vv \neq \ul{0}$, and $\uv$ and $\vv$ are not on the same line.

\begin{center}
\begin{tikzpicture}[scale=0.75]
\draw[->] (-3,0) -- coordinate (x axis mid) (8,0) node[anchor=west]{$x_1$};

\draw[->] (0,-0.5) -- coordinate (y axis mid) (0,4) node[anchor=south]{$x_2$};;
    	%ticks

% actual plane...
\fill[gray, opacity=0.2] (-1.5, -1) rectangle (6, 3);		

% vector u
\draw[->,thick] (0,0)--(3,1);
% label for u
\draw (3, 0.2) node[anchor=south] {$\uv$};

% vector v
\draw[->,thick] (0,0)--(1,2);
% label for v
\draw (1.1, 1.8) node[anchor=west] {$\vv$};
\end{tikzpicture}
\end{center}

\item A \textbf{line}, if $\uv$ and $\vv$ are on the same line.
\begin{center}
\begin{tikzpicture}[scale=0.75]
\draw[->] (-2,0) -- coordinate (x axis mid) (8,0) node[anchor=west]{$x_1$};
\draw[->] (0,-0.5) -- coordinate (y axis mid) (0,4) node[anchor=south]{$x_2$};;
    	%ticks

% line defined by u
\draw[red] (-3,-1)--(9,3);

% vector v
\draw[->,thick] (0,0)--(4, 1.34);
% label for v
\draw (3.7,0.8) node[anchor=west] {$\vv$};

% vector u
\draw[->,thick] (0,0)--(3,1);
% label for u
\draw (2.9, 0.3) node[anchor=south] {$\uv$};
\end{tikzpicture}
\end{center}

\begin{note}
Consider $\uv,\vv\in\R^n$. $\uv$ and $\vv$ are on the same line if there exists scalars $\alpha$ and $\beta$ such that $\alpha\uv + \beta\vv = \ul{0}$, when $\alpha$ and $\beta\not=0$ 
\end{note}

\item A \textbf{point}, if $\uv=\ul{0}$ and $\vv=\ul{0}\Rightarrow\alpha\uv+\beta\vv = \ul{0}$ 

\end{enumerate}

\end{example}

\section{Dot Product (Scalar product)}

\begin{definition}
Let us consider two vectors $\uv\in\R^n$ and $\vv\in\R^n$. The dot (or scalar) product of vectors $\uv$ and $\vv$ is defined as 
\[
\langle\uv,\vv\rangle = u_1v_1+u_2v_2+\dots+{u}_n{v}_n = \sum\limits^{n}_{i=1}u_iv_i
\]
\end{definition}

\begin{notation}
We will use $\langle\uv,\vv\rangle$ to denote the dot product, but sometimes $\uv\cdot\vv$ is used.
\end{notation}
\begin{example}
\[
\uv = \colvec{3}{1}{-1}{3}, \vv = \colvec{3}{0}{\frac{1}{2}}{-1}
\]
\[
\langle \uv,\vv\rangle = 1\cdot 0 + (-1)\cdot\frac{1}{2} + 3\cdot (-1) = -3.5
\]
\end{example}

\begin{example}
\[
\uv = \colvec{2}{1}{0}, \vv = \colvec{3}{0}{1}{1}
\]
\[
\langle\uv,\vv\rangle = 0
\]
\end{example}

\subsection{Properties of Dot Product}
\begin{enumerate}
	\item $\langle \alpha\cdot\uv,\vv\rangle = \alpha\cdot\langle\uv,\vv\rangle$ for any $\alpha\in\R,\uv\in\R^n,\vv\in\R^n$. 
	\begin{proof}
\[
\langle \alpha\cdot\uv,\vv\rangle = (\alpha u_1)\cdot v_1 + \dots + (\alpha u_n)\cdot v_n\\
= \alpha\cdot(u_1\cdot v_1+\dots+u_n\cdot v_n)\\
= \alpha\cdot\langle\uv,\vv\rangle
\]	
	\end{proof}	
		
\item $\langle\uv,\alpha\vv\rangle = \alpha\langle\uv,\vv\rangle$ for any $\alpha\in\R,\uv,\vv\in\R^n$
\item $\langle\alpha\uv+\beta\vv,\wv\rangle = \alpha\cdot\langle\uv,\wv\rangle + \beta\langle\vv,\wv\rangle, \forall\alpha\in\R,\forall\uv,\vv,\wv\in\R^n$
\end{enumerate}
\begin{example}
Let us consider $\uv = \colvec{2}{3}{4}. \langle \uv,\uv\rangle = 3\cdot 3+4\cdot 4 = 9+16 = 25 = 5^2$
\begin{center}
\begin{tikzpicture}[scale=0.5]
\draw[->] (-4,-2) -- coordinate (x axis mid) (4,-2) node[anchor=west]{$x_1$};
\draw[->] (-2,-4) -- coordinate (y axis mid) (-2,4) node[anchor=south]{$x_2$};
\draw[->](-2,-2)--(1,2) node[anchor=west,yshift=0.15cm]{$\uv = \colvec{2}{3}{4}$};
\draw[dashed](1,-2)--(1,2);
\draw [decorate,decoration={brace,amplitude=10pt,mirror,raise=0pt},yshift=0pt]
(1,-2) -- (1,2) node [anchor=west,black,midway,xshift=0.3cm] {4};
\draw [decorate,decoration={brace,amplitude=10pt,mirror,raise=0pt},yshift=0pt]
(-2,-2) -- (1,-2) node [anchor=north,black,midway,yshift=-0.3cm] {3};
\end{tikzpicture}
\end{center}
\end{example}

\section{Length of a Vector}
\begin{definition}
The length of vector $\uv\in\R^n, \norm{\uv}$, is defined as $\norm{\uv} = \sqrt{\langle\uv,\uv\rangle}$. Sometimes it is also called the Euclidian norm of $\uv$.
\end{definition}

\section{Unit Vectors}
\begin{definition}
A vector with length equal to 1 is called a unit vector
\end{definition}

\subsection{How to Normalize a Vector?}
Sometimes we just want to deal with unit vectors, because some calculations might be simpler. \\
If we take the vector $\uv\not=\ul{0}$, how do we make it a \textit{unit vector}, or, in other words, how do we \textit{normalise} it? \\
To normalise a vector $\uv$, we simply have to multiply $\uv$ by the inverse of its length, or $\frac{1}{\norm{\uv}}$, and we obtain
$$\ul{\hat{u}} = \frac{1}{\norm{\uv}}\cdot \ul{u} = \frac{\uv}{\norm{\uv}}$$
\begin{note}
$\ul{\hat{u}}$ is \textbf{not} $\ul{u}$.
\end{note}

\begin{notation}
Usually unit vectors are denoted with an \textit{hat} over them, like $\ul{\hat{v}}$.
\end{notation}

\begin{example}
\noindent Consider the following vector
\[
\uv = \colvec{2}{3}{4}
\]
The \textit{unit vector} $\ul{\hat{v}}$ with the same direction of $\uv$ is then 
\[\ul{\hat{v}}=\frac{\uv}{\norm{\uv}} = \frac{1}{5}\cdot \colvec{2}{3}{4} = \colvec{2}{\frac{3}{5}}{\frac{4}{5}} = \colvec{2}{0.6}{0.8}\]
\end{example}
\begin{note}
You can check by yourself if $\ul{\hat{v}}$ is really a unit vector by calculating its length.
\end{note}

\section{Angle between Vectors}
Let us consider $\R^2$. What is the set of all possible endpoints of unit vectors (vectors of length $1$, usually denoted with an hat above them $\hat{u}$) in $\R^2$, originating from the origin? A unit circle or, in other words, a circle with radius $1$.

\adjustbox{valign=t}{
\begin{minipage}[t]{0.45\linewidth}
\begin{center}
\begin{tikzpicture}[scale=0.5]
\draw[->] (-4,0) -- coordinate (x axis mid) (4,0) node[anchor=west]{$x_1$};
\draw[->] (0,-4) -- coordinate (y axis mid) (0,4) node[anchor=south]{$x_2$};
\draw[] (0,0) circle (3);
\draw[] (0,0)--(1.665,2.5);
\draw (0.5,0) arc (0:57:0.5);
\draw (0.5,0) node[anchor=south west]{$\theta$};
\draw[dashed](1.67,2.5) -- (1.67,0) node[anchor=north]{$u_1$};
\draw[dashed](1.67,2.5) -- (0,2.5)node[anchor=east]{$u_2$};
\draw (0.86,1.25) node[anchor=east]{$\uv$};
\draw[color=white] (5,0) -- (6.8,0); % Filler to align the images
\end{tikzpicture}
\end{center}
\end{minipage}}
\hfill
\adjustbox{valign=t}{
\begin{minipage}[t]{0.45\linewidth}
\begin{align*}
\uv &= \colvec{2}{u_1}{u_2}\\
\cos(\theta) &= \frac{u_1}{\norm{\uv}} = u_1\\
\sin(\theta) &= \frac{u_2}{\norm{\uv}} = u_2\\
\uv &= \colvec{2}{\cos(\theta)}{\sin(\theta)}
\end{align*}
\end{minipage}}\\

\subsection{Angle between Unit Vectors}
\noindent Let us consider two \textit{unit vectors}

\adjustbox{valign=t}{
\begin{minipage}[t]{0.45\linewidth}
\begin{center}
\begin{tikzpicture}[scale=0.5]
\draw[->] (-4,0) -- coordinate (x axis mid) (4,0) node[anchor=west]{$x_1$};
\draw[->] (0,-4) -- coordinate (y axis mid) (0,4) node[anchor=south]{$x_2$};
\draw[] (0,0) circle (3);
\draw[->] (0,0)--(2.82,1.1) node[anchor=west] {$\ul{u} = \colvec{2}{\cos\theta}{\sin\theta}$};
\draw[->] (0,0)--(1.55,2.6)node[anchor=south west] {$\ul{v} = \colvec{2}{\cos\varphi}{\sin\varphi}$};
\draw (1.5,0) arc (0:21:1.5);
\draw (1.2,0.45) arc (21:61:1.2);
\draw (2.2,0) arc (0:59:2.2);
\draw(1.75,0.4) node{$\theta$};
\draw(1.2,1) node{$\psi$};
\draw(2,1.5) node{$\varphi$};
\end{tikzpicture}
\end{center}
\end{minipage}
}
\hfill
\adjustbox{valign=t}{
\begin{minipage}[t]{0.45\linewidth}
\begin{align*}
\langle\uv,\vv\rangle &= \cos(\theta)\cos(\varphi) + \sin(\theta)\sin(\varphi)\\
&= \cos(\theta- \varphi) = \cos(\psi)\\
&= \cos(\angle(\uv,\vv))
\end{align*}
\end{minipage}
}

\subsection{Angle between Non-Unit Vectors}
If $\uv \not=\ul{0}$ or $\vv \not=\ul{0}$ are \textbf{not} unit vectors, we can find the angle between them in the following way
\[
\langle\uv,\vv\rangle = \left\langle \norm{\uv}\cdot\frac{1}{\norm{\uv}} \cdot \uv,\norm{\vv}\cdot\frac{1}{\norm{\vv}} \cdot \vv\right\rangle\\
= \norm{\uv}\norm{\vv}\left\langle \underbrace{\frac{1}{\norm{\uv}} \cdot \uv,\frac{1}{\norm{\vv}} \cdot \vv}_{\text{Unit Vectors}}\right\rangle\\ 
\]

\[
\langle\uv,\vv\rangle = \norm{\uv}\norm{\vv}\cos\left( \angle(\uv,\vv)\right)
\]

\begin{lemma}
If $\uv\not=\ul{0},\vv\not=\ul{0}, \uv\in\R^n,\vv\in\R^n$, then 
\[\cos\left(\angle(\uv,\vv) \right) = \frac{\langle \uv,\vv\rangle}{\norm{\uv}\norm{\vv}}\]
\end{lemma}

\section{Cauchy – Schwarz inequality}
\begin{lemma}
The \textit{Cauchy–-Schwarz inequality} is a useful inequality encountered in many different settings, such as \textit{linear algebra}, and it states that
for any $\uv\in\R^n$ and $\vv\in\R^n$
\[\abs{\langle\uv,\vv\rangle}\leq\norm{\uv}\norm{\vv}\]
\end{lemma}
\begin{remark}
It is easy to see that \textit{Cauchy-–Schwarz inequality} is also correct for zero vectors.	
\end{remark}

\begin{note}
In the previous section, we have seen that 
$$\langle\uv,\vv\rangle = \norm{\uv}\norm{\vv}\cdot\cos\left( \angle(\uv,\vv)\right)$$
Now, let us take the absolute value of both sides
\[\abs{\langle\uv,\vv\rangle} = \abs{\norm{\uv}\norm{\vv}\cdot\abs{\cos\left( \angle(\uv,\vv)\right)}}=\norm{\uv}\norm{\vv}\cdot\abs{\cos\left( \angle(\uv,\vv)\right)}
\]
Notice that $\abs{\cos\left( \angle(\uv,\vv)\right)}\leq 1$.
\end{note}